\documentclass{article}

\usepackage{fancyhdr}
\usepackage{amsmath,amssymb,amsthm}
\usepackage[utf8]{inputenc}
\usepackage[T1]{fontenc}
\usepackage{enumitem}
\pagestyle{fancy}
\lhead{An introduction to the analysis of algorithms\break Michael Soltys}
\rhead{Problem 2.9 Solution\break Ryan McIntyre}
\renewcommand{\headrulewidth}{0.4pt}
\renewcommand{\headheight}{24pt}
\newcommand{\opt}{\text{\sc Opt}}
\pagenumbering{gobble}

\newtheorem{inner}{Problem}
\newenvironment{prb}[1]
	{\renewcommand\theinner{#1}\inner}
	{\endinner}

\newenvironment{solution}
	{\noindent{\bf Solution:}}{\hfill$\square$}

\begin{document}

\begin{prb}{2.9}
Suppose that $G=(V,E)$ is not connected. Show that in this case, when
$G$ is given to Kruskal's algorithm as input, the algorithm computs a
spanning forest of $G$. First, precisely define a connected component
and a spanning forest.
\end{prb}

\begin{solution}
Given an undirected graph $G=(V,E)$, a connected component $C=(V_c,E_c)$
of $G$ is a nonempty subset $V'$ of $V$ (along with its included edges) 
such that for all pairs of vertices $u,v\in V'$, there is a path
from $u$ to $v$ (which we'll state as ``$u$ and $v$ are connected''),
and moreover for all pairs of vertices $x,y$ such that $x\in V'$ and
$y\in V-V'$, $x$ and $y$ are {\emph not} connected (i.e. there is no
path from $x$ to $y$).
We can make a few quick obvservations about connected components:
\begin{enumerate}[label=(\roman*)]
	\item The connected components of any graph comprise a partition of
		its edge and vertex sets, as connectedness is an equivalence relation.
	\item Given any edge, both of its endpoints are in the same component,
		as it defines a path connecting them.
	\item Given any two vertices in a connected component, there is a path
		connecting them. Similarly, any two vertices in different components
		are necessarily not connected.
	\item Given any path, every contained edge is in the same component.
\end{enumerate}

A spanning forest is a collection of spanning trees - one for each
connected component. That is, an edge set $F\subseteq E$ is a spanning
forest of $G=(V,E)$ if and only if:
\begin{enumerate}[label=(\roman*)]
	\item $F$ contains no cycles.
	\item $(\forall u,v\in V)$, $F$ connects $u$ and $v$ if and only if
		$u$ and $v$ are connected in $G$.
\end{enumerate}

Let $G=(V,E)$ be a graph that is not connected. That is, $G$ has $>1$ 
components. Let $T_i$ denote the state of $T$, in Kruskal's, after $i$
iterations. Let $C=(V_c,E_c)$ be a component of $G$.
We will use the following loop invariant as proof
that Kruskal's results in a spanning forest for $G$:
\begin{equation}\label{eq1}
\textit{\emph{The edge set $T_i\cup\{e_{i+1},\ldots,e_m\}$ connects all
nodes in $V_c$}}
\end{equation}

The basis case clearly works; $T_0\cup\{e_1,\ldots,e_m\}=E$. Every
vertex in $V_c$ is connected in $G$, and we have every edge in $G$ at our
disposal.

Assume that $T_{i-1}\cup\{e_i,\ldots,e_m\}$ connects all nodes in $V_c$.

\noindent{\bf Case 1:} $e_i$ is not in $E_c$. Clearly $e_i$ has no
effect on the connectedness of $V_c$, as any path in $C$ must be
a subset of $E_c$.

\noindent{\bf Case 2:} $e_i\in E_c$ and $T_{i-1}\cup\{e_i\}$ contains
a cycle. Let $u,v$ be nodes adjacent to $e_i$. $T_{i-1}$ does not 
contain a cycle by construction, so $e_i$ completes a cycle in 
$T_{i-1}\cup\{e_i\}$. Thus, there is already a path $(u,v)$ in $T_{i-1}$,
which can be used to replace $e_i$ in any other path. Therefore,
$T_{i-1}\cup\{e_{i+1},\ldots,e_m\}$ connects everything that was connected
by $T_{i-1}\cup\{e_i,\ldots,e_m\}$, so the assignment of $T_i=T_{i-1}$ works.

\noindent{\bf Case 3:} $e_i\in E_c$ and $T_{i-1}\cup\{e_i\}$ does not
contain a cycle. Then $T_i=T_{i-1}\cup\{e_i\}$, so
$T_i\cup\{e_{i+1},\ldots,e_m\}=T_{i-1}\cup\{e_i,\ldots,e_m\}$, so the loop
invariant holds.

We have shown through induction that the loop invariant (\ref{eq1}) holds.
Note that $C$ was an arbitrary connected component, so $T_m$ for each 
component $C=(V_c,E_c)$ in $G$, $T_m$ connects every node in $V_c$. 
Obviously, if any two nodes in $V$ are not connected in $G$, $T$ does
not connect them; doing so would require edges not in $E$. Therefore,
$T_m$ meets both conditions imposed on a spanning forest above.
\end{solution}

\end{document}
